\chapter{Concluding Remarks}
\label{chap:conculusion}

\section{Conclusion}
In this study, we built a complete pipeline composed of preprocessing, training, evaluation and explanation to detect dementia from raw MRI scans. The models obtained by training on the OASIS dataset did not attain state-of-the-art performances but have the advantage of providing not only a diagnostic but an explanation about which region of the MRI made the model do such a prediction. In addition to that, we built a responsive and easy-to-use web application in react\footnotemark{} that allows any clinician without any machine learning background to quickly gain insight into the prediction our model made. Thus reducing the black box image from which machine learning is suffering from in the field.  
\footnotetext{\href{https://reactjs.org/}{https://reactjs.org/}}

\section{Future Work}
Despite the already interesting results and helpful insight on dementia gained by the visualization, this project has been working with a small amount of labeled data. To have even more interesting results, the entire pipeline would have to be applied to a much larger dataset for which the labels would be provided.

We would also like to try the pipeline with some other modalities instead of using the T1 only. For example, it has been shown that the iron density inside a brain is correlated with dementia\footnote{\href{https://www.medicalnewstoday.com/articles/measuring-iron-in-the-brain-can-point-to-dementia}{https://www.medicalnewstoday.com/articles/measuring-iron-in-the-brain-can-point-to-dementia}}. In fact, we think that the cause of dementia might be invisible through a T1 weighted image and that we are might currently only look at the consequences.